\chapter{Introduction}
\label{chapterlabel1}

\begin{itemize}
    \item Motivation / context.
    \item Explain things more.
    \item Assume that readers know less.
\end{itemize}

\begin{itemize}
    \item Effective pressure
    \item Interaction with rock matrix and pressurised fluids
    \item Poroelastic background
    \item
\end{itemize}


\section{Background}
\subsection{Poroelastic theory}
\subsection{Small section: Applications to post-seismic pore pressure response in fault zones}

\section{Literature review}
Previous measurements of poroelasticity, particularly anisotropic measurements.

\subsection{Effective medium theories}
\subsubsection{Non-interactive}
\begin{itemize}
    \item I.e. Sayers and Kachanov (1995)
    \item What is the general idea? Cracks account for additional compliance in addition to the compliance of the solid material.
    \item What is the non-interactive assumption? And how valid is it? i.e. stress amplification and shielding. Does it break down at high crack densities?
\end{itemize}
\subsubsection{Other schemes}
\begin{itemize}
    \item Mori-Tanaka
    \item Kuater-Toksoz
    \item Differential??
\end{itemize}

\subsection{Comparisons of dynamic and static elastic moduli}
\begin{itemize}
    \item Static measurements involve mechanical changes in stress that vary strain (for example). But these changes are not so small, they also change the microstructual state of the material, i.e. progressive crack closure.
    \item In sandstones there can be a huge difference in moduli.
    \item Why? Microstructure of grain contacts can change significantly due to changes in load.
    \item Not so bad for granites.
\end{itemize}

\subsection{Brittle behaviour and dilatency}


\section{Overview}
What I will talk about in each section.